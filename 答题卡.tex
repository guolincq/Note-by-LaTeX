\documentclass[11pt,space]{ctexart} % ans
\usepackage{GEEexam}

%\watermark{60}{6}{14-金融工程-白兔兔} %水印
\begin{document}\zihao{5}
\juemi %输出绝密
\biaoti{2021年普通高等学校招生全国统一考试}
\fubiaoti{数学B卷}
{\heiti 注意事项}:
\begin{enumerate}[itemsep=-0.3em,topsep=0pt]
\item 答卷前,考生务必将自己的姓名和准考证号填写在答题卡上。
\item 回答选择题时,选出每小题答案后,用铅笔把答题卡对应题目的答案标号涂黑。如需改动,用橡皮擦干净后,再选涂其它答案标号。回答非选择题时,将答案写在答题卡上。写在本试卷上无效。
\item 考试结束后,将本试卷和答题卡一并交回。
请认真核对监考员在答题卡上所粘贴的条形码上的姓名、准考证号与您本人是否相符。	
\end{enumerate}
%%====================================================================
%%—————————————————————————————正文开始———————————————————————————————
%%====================================================================

\section{选择题:本大题共8小题,每小题5分,共40分。在每小题给出的四个选项中,只有一项是符合题目要求的。}

\begin{enumerate}[itemsep=0.3em,topsep=0pt]
	\item  设集合 $ A=\{x \mid-2<x<4\}$,  $B=\{2,3,4,5\} $, 则  $A \cap B=$
	\begin{tasks}(4)
	\task $\{2\} $ \task $\{2,3\}  $ \task $\{3,4\}  $ \task $ \{2,3,4\} $
	\end{tasks}
	
	\item  已知  $z=2-\mathrm{i} $, 则  $z(\bar{z}+\mathrm{i})=$
	\begin{tasks}(4)
	\task $6-2 \mathrm{i} $ \task $4-2 \mathrm{i} $ \task $6+2 \mathrm{i}  $ \task $ 4+2 \mathrm{i} $
	\end{tasks}
	
	\item  已知圆雉的底面半径为 $ \sqrt{2}  $,其侧面展开图为一个半圆,则该圆雉的母线长为
	\begin{tasks}(4)
	\task $ 2$ \task $ 2 \sqrt{2} $ \task $4 $ \task $4 \sqrt{2}  $
	\end{tasks}
	
	\item  下列区间中,函数  $f(x)=7 \sin \left(x-\frac{\pi}{6}\right)$  单调递增的区间是
	
	\begin{tasks}(4)
	\task $ \left(0, \frac{\pi}{2}\right) $ \task $\left(\frac{\pi}{2}, \pi\right) $ \task $ \left(\pi, \frac{3 \pi}{2}\right) $ \task $  \left(\frac{3 \pi}{2}, 2 \pi\right) $
	\end{tasks}

	\item  已知  $F_{1}$, $F_{2}$  是敕圆  $C:$ $\frac{x^{2}}{9}+\frac{y^{2}}{4}=1$  的两个焦点,点 $ M$  在  $C$  上, 则  $\left|M F_{1}\right| \cdot\left|M F_{2}\right|$  的最
	大值为

	\begin{tasks}(4)
	\task $ 13$ \task $ 12 $ \task $ 9$ \task $6 $
	\end{tasks}

	\item   若  $\tan \theta=-2 $, 则 $ \frac{\sin \theta(1+\sin 2 \theta)}{\sin \theta+\cos \theta}=$
	\begin{tasks}(4)
	\task $ -\frac{6}{5}$ \task $-\frac{2}{5} $ \task $\frac{2}{5}  $ \task $\frac{6}{5}  $
	\end{tasks}

	\item    若过点 $ (a, b) $ 可以作曲线 $ y=\mathrm{e}^{x} $ 的两条切线,则
	\begin{tasks}(4)
	\task $ \mathrm{e}^{b}<a$ \task $ \mathrm{e}^{a}<b $ \task $ 0<a<\mathrm{e}^{b}$ \task $ 0<b<\mathrm{e}^{\mathrm{a}} $
	\end{tasks}
	\item   有6个相同的球,分别标有数字1, 2, 3, 4, 5, 6, 从中有放回的随机取两次,
	每次取 1 个球. 甲表示事件“第一次取出的球的数字是 1”, 乙表示虽件“第二次取
	出的球的数字是 2”, 丙表示事件 “两次取出的球的数字之和是 8”, 丁表示虾件 “两
	次取出的球的数字之和是 7”,则
	\begin{tasks}(2)
	\task 甲与丙相互独立 \task 甲与丁相互独立 \task 乙与丙相互独立 \task 丙与丁相互独立
	\end{tasks}
	
\end{enumerate}

\section{选择题:本大题共4小题,每小题5分,共20分。在每小题给出的四个选项中,有多项是符合题目要求的。全部选对得5分,部分选对得2分,有选错的得0分。}


\begin{enumerate}[itemsep=0.3em,topsep=0pt]
	\setcounter{enumi}{8}

	\item    有一组样本数据 $ x_{1}$, $x_{2}$, $\cdots$, $x_{n} $, 由这组数据得到新样本数据  $y_{1}, y_{2}, \cdots, y_{n} $, 其中
	$y_{t}=x_{i}+c(i=1,2, \cdots, n)$, $c $ 为非零常数,则
	\begin{tasks}(2)
	\task  两组样本数据的样本平均数相同
	\task 两组样本数据的样本中位数相同
	\task 两组样本数据的样本标准差相同
	\task 两组样本数据的样本极差相同
	\end{tasks}

\newpage
   \item    已知 $ O $ 为坐标原点, 点 $ P_{1}(\cos \alpha, \sin \alpha)$,$ P_{2}(\cos \beta,-\sin \beta)$, $P_{3}(\cos (\alpha+\beta)$, $\sin (\alpha+\beta)) $,
	$A(1,0) $, 则

	\begin{tasks}(2)
	\task $ \left|\overrightarrow{O P}_{1}\right|=\left|\overrightarrow{O P_{2}}\right| $ \task $ |\overrightarrow{A P}|=\left|\overrightarrow{A P_{2}}\right| $
	\task $\overrightarrow{O A} \cdot \overrightarrow{O P}_{3}=\overrightarrow{O P}_{1} \cdot \overrightarrow{O P_{2}}  $
	\task $\overrightarrow{O A} \cdot \overrightarrow{O P_{1}}=\overrightarrow{O P_{2}} \cdot \overrightarrow{O P_{3}} $
	\end{tasks}

   \item    已知点 $ P $ 在圆 $ (x-5)^{2}+(y-5)^{2}=16 $ 上,点 $ A(4,0)$, $ B(0,2) $, 则
   \begin{tasks}(2)
   \task 点 $ P$  到直线  $A B $ 的距离小于 $10$
   \task 点 $ P $ 到直线 $ A B $ 的距离大于$2$
   \task 当 $ \angle P B A $ 趣小时, $ |P B|=3 \sqrt{2}$
   \task 当  $\angle P B A $ 最大时, $ |P B|=3 \sqrt{2}$
   \end{tasks}

   \item    在正三麥柱 $ A B C-A_{1} B_{1} C_{1} $ 中, $ A B=A A_{1}=1 $, 点 $ P $ 满足  $\overrightarrow{B P}=\lambda \overrightarrow{B C}+\mu \overrightarrow{B B_{1}} $, 其中
	$\lambda \in[0,1]$, $ \mu \in[0,1]$,   则

\begin{tasks}(1)
\task 当 $ \lambda=1$  时, $ \triangle A B_{1} P $ 的周长为定值
\task 当 $ \mu=1  $ 时,三棱雉 $ P-A_{1} B C $ 的体积为定值
\task 当 $ \lambda=\frac{1}{2} $ 时,有且仅有一个点 $ P $, 使得  $A_{1} P \perp B P$
\task 当 $ \mu=\frac{1}{2} $ 时,有且仅有一个点 $ P $, 使得 $ A_{1} B \perp$  平面 $ A B_{1} P$
\end{tasks}

\end{enumerate}

\section{填空题:本题共4小题,每小题5分,共20分。}
\begin{enumerate}[itemsep=0.3em,topsep=0pt,resume]%\setcounter{enumi}{12}

\item  已知函数 $ f(x)=x^{3}\left(a \cdot 2^{x}-2^{-x}\right) $ 是偶函数,则 $ a= $ \blank{}
\item  已知 $ O$  为坐标原点,抛物线 $ C$:$ y^{2}=2 p x(p>0)$  的焦点为 $ F$, $P $ 为 $ C$  上一点, $ P F$  与 $x$  轴垂直, $ Q $ 为 $  x $ 轴上一点,且 $ P Q \perp O P $.  若 $ |F Q|=6 $, 则 $ C $ 的难线方程为 \blank{}
\item  函数 $ f(x)=|2 x-1|-2 \ln x$  的最小值为 \blank{}
\item  某校学生在研究民间秒纸艺术时,发现五纸时经常会沿纸的某条对称轴把纸对折. 规
格为  20 dm $\times$ 12 dm  的长方形纸,对折 1 次共可以得到  10 dm $\times$ 12 dm, 20 dm $\times$ 6 dm  两
种规格的图形,它们的面积之和  $S_{1}=240$ dm$^{2} $, 对折 $2$ 次共可以得到  5 dm $\times$ 12 dm,
	10 dm $\times$ 6 dm, 20 dm $\times$ 3 dm  三种规格的图形,它们的面积之和  $S_{2}=180$ dm$^{2} $, 以此类
推. 则对折 $4$ 次共可以得到不同规格图形的种数为 \blank{}; 如果对折 $ n $ 次,那
么 $ \sum\limits_{k=1}^{n} S_{k}=$ \blank{} m$^{2}$ .

\end{enumerate}

\section{解答题:本题共6小题,共70分。解答应写出文字说明、证明过程或演算步骤。}
\subsection{必考题:60分。}

\begin{enumerate}[itemsep=0.5em,topsep=5pt,resume]%\setcounter{enumi}{17}
	\item (12 分)\\
	 已知数列 $ \left\{a_{n}\right\} $ 满足 $ a_{1}=1$,
	 $a_{n+1}=
	 \begin{cases}
	 a_{n}+1, & n \text { 为奇数 } \\ a_{n}+2, & \text { n为偶数 }
	 \end{cases}$.

	 \begin{enumerate}[itemsep=-0.3em,label={(\arabic*)},topsep=0pt,labelsep=.5em,leftmargin=3em]
		\item 记 $ b_{n}=a_{2 n} $, 写出 $ b_{1}$, $b_{2} $, 并求数列 $ \left\{b_{n}\right\} $ 的通项公式;
		\item 求 $ \left\{a_{n}\right\} $ 的前 20 项和.
	\end{enumerate}

\item  (12 分)\\
某学校组织 “一带一路”知识竟赛,有 A, B  两类问题. 每位参加比赛的同学先在
两类问题中选择一类并从中随机抽取一个问题回答,若回答错误则该同学比赛结束; 若
回答正确则从另一类问题中再随机抽取一个问题回答,无论回答正确与否,该同学比赛
结東. A 类问颈中的每个问题回答正确得 20 分,否则得 0 分:  B  类问题中的每个问题
回答正确得 $80$ 分,否则得 0 分.
已知小月能正确回答  A  类问题的概率为 $ 0.8  $,能正确回答  B  类问题的概率为  0.6 ,
且能正确回答问题的概率与回答次序无关.
\begin{enumerate}[itemsep=-0.3em,label={(\arabic*)},topsep=0pt,labelsep=.5em,leftmargin=3em]
	\item 若小明先回答  A  类问题,记 $ X $ 为小明的累计得分,求 $ X $ 的分布列;
	\item 为使累计得分的期望最大,小明应选择先回答哪类问硕? 并说月理由.
\end{enumerate}

\newpage
\item (12 分)\\
记  $\triangle A B C$  的内角  $A, B, C$  的对追分别为 $ a, b, c $.  已知 $ b^{2}=a c $, 点  $D $ 在边 $ A C$
上. $ B D \sin \angle A B C=a \sin C$.

\begin{enumerate}[itemsep=-0.3em,label={(\arabic*)},topsep=0pt,labelsep=.5em,leftmargin=3em]
	\item 证月: $ B D=b $;
	\item 若 $ A D=2 D C $, 求 $ \cos \angle A B C $.
\end{enumerate}


\item  (12 分)\\
\begin{minipage}[h][20ex][t]{.25\textwidth}
如图, 在三棱雉 $ A-B C D $ 中,平面 $ A B D \perp $ 平面
 $B C D$, $A B=A D$, $O $ 为  $B D$  的中点.

 \begin{enumerate}[itemsep=-0.3em,label={(\arabic*)},topsep=0pt,labelsep=.5em,leftmargin=3em]
	\item 证明:  $O A \perp C D  $;
	\item 若  $\triangle O C D $ 是边长为 1 的等边三角形, 点 $ E $ 在
	棱  $A D  $ 上,  $D E=2 E A $, 且二面角 $ E-B C-D $ 的大小为
	 $45^{\circ} $, 求三棱雉 $ A-B C D $ 的体积.
\end{enumerate}\end{minipage}
	\begin{minipage}[h][30ex][t]{.25\textwidth}
		\includegraphics[width=6.5cm]{21titu.pdf}
	\end{minipage}\vspace{3em}
\newpage
\item ( 12 分) \\
在平面直角坐标系 $ x O y $ 中,已知点 $ F_{1}(-\sqrt{17}, 0)$, $F_{2}(\sqrt{17}, 0)$,  点 $ M $ 满足
$ \left|M F_{1}\right|-\left|M F_{2}\right|=2 $.  记 $ M$  的轨迹为  $C$.
\begin{enumerate}[itemsep=-0.3em,label={(\arabic*)},topsep=0pt,labelsep=.5em,leftmargin=3em]
	\item 求 $ C$  的方程;
	\item 设点 $ T$  在直线 $ x=\frac{1}{2} $ 上, 过 $ T $ 的两条直线分别交 $ C$  于  $A$, $B $ 两点和 $ P$, $Q $ 两点,
	且  $|T A| \cdot|T B|=|T P| \cdot|T Q| $, 求直线  $A B $ 的斜率与直线 $ P Q $ 的斜率之和.
\end{enumerate}


\item (12 分)\\
已知函数 $ f(x)=x(1-\ln x) $.

\begin{enumerate}[itemsep=-0.3em,label={(\arabic*)},topsep=0pt,labelsep=.5em,leftmargin=3em]
	\item 讨论 $ f(x) $ 的单调性;
	\item 设 $ a$, $b $ 为两个不相等的正数,且 $ b \ln a-a \ln b=a-b $, 证明: $ 2<\frac{1}{a}+\frac{1}{b}<\mathrm{e} $.
\end{enumerate}

\end{enumerate}




%%%%%%%%%%%%%%%%%%%%%%%%%%%%%%%%%%%%%%%%%%%%%%%%%%%%%%%%%%%%%%%%%%%%%%%%%%%%%%%
%------------------------------------结束--------------------------------------
%%%%%%%%%%%%%%%%%%%%%%%%%%%%%%%%%%%%%%%%%%%%%%%%%%%%%%%%%%%%%%%%%%%%%%%%%%%%%%%
\clearpage	
	
\end{document}

%合并文档一面双页
\documentclass{article}
\usepackage{pdfpages}
\usepackage[paperwidth=39.5cm,paperheight=27.2cm]{geometry}
\begin{document}
	\includepdf[pages=1-6,nup=2x1]{dibajiefeishujuesaijuanzi.pdf}
\end{document}
doublepages
